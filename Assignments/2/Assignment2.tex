\documentclass{article}

\usepackage[tmargin=0.5in,bmargin=0.25in]{geometry}
\usepackage{amsmath, amssymb, amsthm}
\usepackage{listings}
\usepackage{multicol}
\usepackage{enumitem}

\title{CSCI 305 Assignment 2}
\author{(Solo) Isaac Boaz}

\begin{document}
\maketitle

\begin{enumerate}
    \item  \begin{enumerate}[label=\arabic*.]
              \item Prove that finite sums and $\Theta$ commute.
                    \begin{equation*}
                        \sum_{i=1}^{n}{\Theta(f(i))} = \Theta(\sum_{i=1}^{n}{f(i)})
                    \end{equation*}
                    \begin{proof}
                        To show that finite sums and $\Theta$ commute, we must show that each has the same inequality constraints. \\[1em]
                        \textbf{Left-Hand Side}.
                        \begin{align*}
                                                                     & \sum_{i=1}^{n}{\Theta(f(i))}                              \\
                            \rightarrow \sum_{i=1}^{n}{c_1f(i)} \leq & \sum_{i=1}^{n}{\Theta(f(i))} \leq \sum_{i=1}^{n}{c_2f(i)} \\
                            \rightarrow c_1\sum_{i=1}^{n}{f(i)} \leq & \sum_{i=1}^{n}{\Theta(f(i))} \leq c_2\sum_{i=1}^{n}{f(i)}
                        \end{align*}
                        \textbf{Right-Hand Side}.
                        \begin{align*}
                                                                     & \Theta(\sum_{i=1}^{n}{f(i)})                              \\
                            \rightarrow c_1\sum_{i=1}^{n}{f(i)} \leq & \Theta(\sum_{i=1}^{n}{f(i)}) \leq c_2\sum_{i=1}^{n}{f(i)}
                        \end{align*}
                        Since both equations have the same lower and upper bounds, $\Theta$ is commutive.
                    \end{proof}
              \item Prove that \(\log_an = \Theta(\lg n)\) for any \(a > 1\), where \(\lg\) is \(\log_2\).
                    \begin{align*}
                        \log_an = \Theta(\lg n) \implies                          \\
                        \exists c_1, c_2, n_0 \in \mathbb{R}^+ \text{ such that } \\
                        c_1\lg n \leq \log_an \leq c_2\lg n \text{ for all } n \geq n_0
                    \end{align*}
                    Note that for any \(a\) or \(n\), we can set \(c_1 = 0\), Leaving us with the right-hand side of the equation.
                    \begin{align*}
                        \log_an                & \leq c_2 \lg n         \\
                        \frac{\log_2n}{log_2a} & \leq c_2 \log_2n       \\
                        c_2                    & \geq \frac{1}{\log_2a}
                    \end{align*}
                    Showing us that this holds true for any \(a > 1\).
                    \pagebreak
              \item Prove that for \(k\) integer, \(\sum_{i=1}^{n}{i^k} = \Theta(n^{k+1})\) \\
                    \begin{align*}
                        \sum_{i=1}^{n}{i^k} = \Theta(n^{k+1}) \implies \\
                        c_1n^{k+1} \leq \sum_{i=1}^{n}{i^k} \leq c_2n^{k+1}
                    \end{align*}
                    Similarly to the previous question, we can set \(c_1 = 0\) to resolve the lower bound.
                    As for the upper bound, notice
                    \begin{align*}
                        \sum_{i=1}^{n}{i^k} & \leq \sum_{i=1}^{n}{n^k} = n \cdot n^k = n^{k+1} \\
                        \sum_{i=1}^{n}{i^k} & \leq n^{k+1}
                    \end{align*}
                    hence \(\sum_{i=1}^{n}{i^k} = O(n^{k+1})\).
          \end{enumerate}
    \item  \begin{enumerate}[label=\arabic*.]
              \item Let \(p > 0\). Show that \(\log n = o(n^p)\). \\
                    To show that \(\log n = o(n^p)\), we must show that \(\lim_{n \to \infty}{\frac{\log n}{n^p}} = 0\).
                    \begin{align*}
                        \lim_{n \to \infty}{\frac{\log n}{n^p}} & = \lim_{n \to \infty}{\frac{\frac{1}{n}}{pn^{p-1}}} \\
                                                               & = \lim_{n \to \infty}{\frac{1}{pn^p}}                \\
                                                               & = 0
                    \end{align*}
          \end{enumerate}
\end{enumerate}


\end{document}