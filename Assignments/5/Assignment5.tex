\documentclass{article}

\usepackage[bmargin=0.25in]{geometry}
\usepackage{amsmath, amssymb, amsthm}
\usepackage{listings}
\usepackage{multicol}
\usepackage{enumitem}
\usepackage{forest}
\usepackage[group-separator={,}]{siunitx}

\title{\vspace*{-3.5em} CSCI 305 Assignment 5}
\author{(Solo) Isaac Boaz}

\begin{document}
\maketitle

\begin{enumerate}
    \item Recreating the code in Python and running for all $x \in [1, 10]$ shows us that $x$ could be $3, 4, 5, 6, 7, 8, 9$.
          We execute 3 swaps for $x \in [3, 7]$, and 4 swaps for $x \in \{8, 9\}$.
    \item Let's first re-create the heap visually.
          \begin{multicols}{2}
              \begin{forest}
                  [17
                          [15
                                  [7
                                          [3]
                                          [2*]
                                  ]
                                  [12
                                          [11]
                                          [,phantom]
                                  ]
                          ]
                          [12
                                  [7]
                                  [2]
                          ]
                  ]
              \end{forest}
              \columnbreak \\
              We can visually see that the parent of the 2* node is 7, so it \textit{definitely} needs to be $> 7$.
              Keeping in mind one swap would put the new value as the child of 15, having another swap would require it
              to be $> 15$. Thus, we should do \textbf{4. Change 2 to 16}.
          \end{multicols}
    \item Assuming \(h(x)\) uniformly distriutes amongst \(m\) slots, the likelihood of any element being inserted into any specific
          slot is \(\frac{1}{m}\). With \(n\) distinct keys, we get
          \begin{equation*}
              \frac{n}{m} \text{ expected collisions}
          \end{equation*}
    \item \begin{enumerate}[label=\arabic*.]
              \item Since $X$ represents the number of comparisons, all possible values of $X$ means $[1, 4]$.
                    Let's make a table for each possible \(x \in X\). \\
                    \begin{tabular}{c|c|c}
                        $x$ & $P(x)$ & $x \cdot P(x)$ \\
                        \hline
                        1   & 1/10   & $0.1$          \\
                        2   & 2/10   & $0.4$          \\
                        3   & 4/10   & $1.2$          \\
                        4   & 3/10   & $1.2$
                    \end{tabular}
              \item Summing up all the values in the last column, we get $E[X] = 2.9$.
              \item Calculating $Pr(K)$ just involves us seeing which unique values from $[1, 30]$ are in the tree.
                    Seeing as there are no duplicates, we know there are \(10\) values in the tree, and thus
                    there is a \(\frac{10}{30} = \frac{1}{3}\) chance of any value (from \([1, 30]\)) being in the tree. \\
                    Thus, \(Pr(K) = \frac{1}{3} \rightarrow Pr(\overline{K}) = 1 - Pr(K) = \frac{2}{3} \rightarrow Pr(K) + Pr(\overline{K}) = 1\).
              \item No. $X$ is \textit{number of comparisons required to find a node \dots \textbf{from the set of nodes in the tree} \dots}.
                    Whereas $Y$ is the case that the key is quered uniformly from \{1, 2, \dots, 30\}. Since $K$ is given, that is simply rephrasing
                    ``what is the probability that we do $y$ comparisons \textbf{given $y$ is in the tree}'', which is the same as $X$. \textbf{ie} \(Pr(Y = y | K) = Pr(X)\).
              \item
                    \begin{multicols}{2}
                        \begin{tabular}{c|c||c|c}
                            Value & Comparisons & Value & Comparisons \\
                            \hline
                            2     & 4           & 20    & 3           \\
                            4     & 3           & 21    & 3           \\
                            5     & 3           & 22    & 3           \\
                            6     & 3           & 23    & 3           \\
                            8     & 4           & 25    & 3           \\
                            10    & 4           & 26    & 3           \\
                            11    & 4           & 26    & 3           \\
                            13    & 4           & 28    & 3           \\
                            15    & 4           & 29    & 3           \\
                            16    & 4           & 30    & 3           \\
                        \end{tabular}
                        \columnbreak \\[5em]
                        Converting this table to an expected value table:
                        \begin{tabular}{c|c|c}
                            $y$ & $Pr(y)$ & $y \cdot Pr(y)$ \\
                            \hline
                            3   & 13/20   & 1.95            \\
                            4   & 7/20    & 1.4
                        \end{tabular}
                    \end{multicols}
              \item \(E[Y | \overline{K}] = 3.35\)
              \item \begin{align*}
                        E[Y] & = E[Y = y | K]Pr(K)      + E[Y = y | \overline{K}]Pr(\overline{K}) \\
                             & = 2.9 \cdot \frac{1}{3}  + 3.35 \cdot \frac{2}{3}                  \\
                             & = 3.2
                    \end{align*}
          \end{enumerate}

\end{enumerate}

\end{document}