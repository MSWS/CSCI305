\documentclass{article}

\usepackage[bmargin=0.25in]{geometry}
\usepackage{amsmath, amssymb, amsthm}
\usepackage{listings}
\usepackage{multicol}
\usepackage{enumerate,enumitem}

\title{\vspace{-1.2in} CSCI 305 HW 6}
\author{Isaac Boaz}

\begin{document}
\maketitle

\begin{equation*}
      X_k = \sum_{n=0}^{N-1} x_n e^{-\frac{2i \pi}{N}kn}
\end{equation*}

\section*{Problem 1}

Compute the DFT of \dots

\begin{enumerate}[label=\Alph*)]
      \item \([0, 1, 0, -1]\)
            \begin{enumerate}[label={k=\arabic*}, start=0]
                  \item \begin{align*}
                              \sum_{n=0}^{3}{x_n e^{-\frac{2 i \pi}{4}0n}} & = \sum_{n=0}^{3}{x_n e^{-\frac{2 i \pi}{4}0n}}                                \\
                                                                           & = 0 + e^{\frac{-2 i \pi}{4} 0 \cdot 1} + 0 - e^{\frac{-2 i \pi}{4} 0 \cdot 3} \\
                                                                           & = 0 + 1 + 0 - 1                                                               \\
                                                                           & = 0
                        \end{align*}
                  \item \begin{align*}
                              \sum_{n=0}^{3}{x_n e^{-\frac{2 i \pi}{4}1n}} & = \sum_{n=0}^{3}{x_n e^{-\frac{2 i \pi}{4}1n}}                                \\
                                                                           & = 0 + e^{\frac{-2 i \pi}{4} 1 \cdot 1} + 0 - e^{\frac{-2 i \pi}{4} 1 \cdot 3} \\
                                                                           & = 0 - i + 0 - i                                                               \\
                                                                           & = 0 -2i
                        \end{align*}
                  \item \begin{align*}
                              \sum_{n=0}^{3}{x_n e^{-\frac{2 i \pi}{4}2n}} & = \sum_{n=0}^{3}{x_n e^{-\frac{2 i \pi}{4}2n}}                                \\
                                                                           & = 0 + e^{\frac{-2 i \pi}{4} 2 \cdot 1} + 0 - e^{\frac{-2 i \pi}{4} 2 \cdot 3} \\
                                                                           & = 0 - 1 + 0 + 1                                                               \\
                                                                           & = 0
                        \end{align*}
                  \item \begin{align*}
                              \sum_{n=0}^{3}{x_n e^{-\frac{2 i \pi}{4}3n}} & = \sum_{n=0}^{3}{x_n e^{-\frac{2 i \pi}{4}3n}}                                \\
                                                                           & = 0 + e^{\frac{-2 i \pi}{4} 3 \cdot 1} + 0 - e^{\frac{-2 i \pi}{4} 3 \cdot 3} \\
                                                                           & = 0 + i + 0 + i                                                               \\
                                                                           & = 0 + 2i
                        \end{align*}
            \end{enumerate}
            \begin{equation*}
                  DFT([0, 1, 0, -1]) = [0, -2i, 0, 2i]
            \end{equation*}
            \textbf{Analysis}: Since the input does not have a low-frequency presence of data, it makes sense that \(\text{DFT}_0\) is 0. As for the complex
            parts of the DFT, I assume it has something to do with the polarity of the signal / frequency.
      \item \pagebreak \([1, 1, 1, 1]\)
            (In-Class Exercise)
            \begin{equation*}
                  DFT([1, 1, 1, 1]) = [4, 0, 0, 0]
            \end{equation*}
            \textbf{Analysis}: We see that the data has a low frequency of data (all 1-s), so it makes sense that the highest level (lowest frequency) of the DFT is 4.
      \item \([0, -1, 0, 1]\)
            \begin{enumerate}[label={k=\arabic*}, start=0]
                  \item \begin{align*}
                              \sum_{n=0}^{3}{x_n e^{-\frac{2 i \pi}{4}0n}} & = \sum_{n=0}^{3}{x_n e^{-\frac{2 i \pi}{4}0n}}                                \\
                                                                           & = 0 - e^{\frac{-2 i \pi}{4} 0 \cdot 1} + 0 + e^{\frac{-2 i \pi}{4} 0 \cdot 3} \\
                                                                           & = 0 - 1 + 0 + 1                                                               \\
                                                                           & = 0
                        \end{align*}
                  \item \begin{align*}
                              \sum_{n=0}^{3}{x_n e^{-\frac{2 i \pi}{4}1n}} & = \sum_{n=0}^{3}{x_n e^{-\frac{2 i \pi}{4}1n}}                                \\
                                                                           & = 0 - e^{\frac{-2 i \pi}{4} 1 \cdot 1} + 0 + e^{\frac{-2 i \pi}{4} 1 \cdot 3} \\
                                                                           & = 0 + i + 0 + i                                                               \\
                                                                           & = 0 + 2i
                        \end{align*}
                  \item \begin{align*}
                              \sum_{n=0}^{3}{x_n e^{-\frac{2 i \pi}{4}2n}} & = \sum_{n=0}^{3}{x_n e^{-\frac{2 i \pi}{4}2n}}                                \\
                                                                           & = 0 - e^{\frac{-2 i \pi}{4} 2 \cdot 1} + 0 + e^{\frac{-2 i \pi}{4} 2 \cdot 3} \\
                                                                           & = 0 + 1 + 0 - 1                                                               \\
                                                                           & = 0
                        \end{align*}
                  \item \begin{align*}
                              \sum_{n=0}^{3}{x_n e^{-\frac{2 i \pi}{4}3n}} & = \sum_{n=0}^{3}{x_n e^{-\frac{2 i \pi}{4}3n}}                                \\
                                                                           & = 0 - e^{\frac{-2 i \pi}{4} 3 \cdot 1} + 0 + e^{\frac{-2 i \pi}{4} 3 \cdot 3} \\
                                                                           & = 0 - i + 0 - i                                                               \\
                                                                           & = 0 - 2i
                        \end{align*}
                        \begin{equation*}
                              DFT([0, -1, 0, 1]) = [0, 2i, 0, -2i]
                        \end{equation*}
                        \textbf{Analysis}: Since the input array is actually the same as (1), it makes sense that the output array would be the same with reversed signs.
            \end{enumerate}
      \item \pagebreak \([0, 1, 0, -1, 0, 1, 0, -1]\)
            \begin{enumerate}[label={k=\arabic*}, start=0]
                  \item \begin{align*}
                              \sum_{n=0}^{7}{x_n e^{-\frac{2 i \pi}{4}0n}} & = \sum_{n=0}^{7}{x_n e^{-\frac{2 i \pi}{8}0n}}                                                                                                              \\
                                                                           & = 0 + e^{\frac{-2 i \pi}{8} 0 \cdot 1} + 0 - e^{\frac{-2 i \pi}{8} 0 \cdot 3} + 0 + e^{\frac{-2 i \pi}{8} 0 \cdot 5} + 0 - e^{\frac{-2 i \pi}{8} 0 \cdot 7} \\
                                                                           & = 0 + 1 + 0 - 1 + 0 + 1 + 0 - 1                                                                                                                             \\
                                                                           & = 0
                        \end{align*}
                  \item \begin{align*}
                              \sum_{n=0}^{7}{x_n e^{-\frac{2 i \pi}{4}1n}} & = \sum_{n=0}^{7}{x_n e^{-\frac{2 i \pi}{8}1n}}                                                                                                              \\
                                                                           & = 0 + e^{\frac{-2 i \pi}{8} 1 \cdot 1} + 0 - e^{\frac{-2 i \pi}{8} 1 \cdot 3} + 0 + e^{\frac{-2 i \pi}{8} 1 \cdot 5} + 0 - e^{\frac{-2 i \pi}{8} 1 \cdot 7} \\
                                                                           & = 0 + e^{-\frac{i \pi}{4}} + 0 - e^{-\frac{3 i \pi}{4}} + 0 + e^{\frac{3 i \pi}{4}} + 0 - e^{\frac{i \pi}{4}}                                               \\
                                                                           & = 0
                        \end{align*}
                  \item \begin{align*}
                              \sum_{n=0}^{7}{x_n e^{-\frac{2 i \pi}{4}2n}} & = \sum_{n=0}^{7}{x_n e^{-\frac{2 i \pi}{8}2n}}                                                                                                              \\
                                                                           & = 0 + e^{\frac{-2 i \pi}{8} 2 \cdot 1} + 0 - e^{\frac{-2 i \pi}{8} 2 \cdot 3} + 0 + e^{\frac{-2 i \pi}{8} 2 \cdot 5} + 0 - e^{\frac{-2 i \pi}{8} 2 \cdot 7} \\
                                                                           & = 0 - i + 0 - i + 0 - i + 0 - i                                                                                                                             \\
                                                                           & = 0 - 4i
                        \end{align*}
                  \item \begin{align*}
                              \sum_{n=0}^{7}{x_n e^{-\frac{2 i \pi}{4}3n}} & = \sum_{n=0}^{7}{x_n e^{-\frac{2 i \pi}{8}3n}}                                                                                                              \\
                                                                           & = 0 + e^{\frac{-2 i \pi}{8} 3 \cdot 1} + 0 - e^{\frac{-2 i \pi}{8} 3 \cdot 3} + 0 + e^{\frac{-2 i \pi}{8} 3 \cdot 5} + 0 - e^{\frac{-2 i \pi}{8} 3 \cdot 7} \\
                                                                           & = 0 + e^{-\frac{3 i \pi}{4}} + 0 - e^{-\frac{i \pi}{4}} + 0 + e^{\frac{i \pi}{4}} + 0 - e^{\frac{3 i \pi}{4}}                                               \\
                                                                           & = 0
                        \end{align*}
                  \item \begin{align*}
                              \sum_{n=0}^{7}{x_n e^{-\frac{2 i \pi}{4}4n}} & = \sum_{n=0}^{7}{x_n e^{-\frac{2 i \pi}{8}4n}}                                                                                                              \\
                                                                           & = 0 + e^{\frac{-2 i \pi}{8} 4 \cdot 1} + 0 - e^{\frac{-2 i \pi}{8} 4 \cdot 3} + 0 + e^{\frac{-2 i \pi}{8} 4 \cdot 5} + 0 - e^{\frac{-2 i \pi}{8} 4 \cdot 7} \\
                                                                           & = 0 + 1 + 0 - 1 + 0 + 1 + 0 - 1                                                                                                                             \\
                                                                           & = 0
                        \end{align*}
                  \item \begin{align*}
                              \sum_{n=0}^{7}{x_n e^{-\frac{2 i \pi}{4}5n}} & = \sum_{n=0}^{7}{x_n e^{-\frac{2 i \pi}{8}5n}}                                                                                                              \\
                                                                           & = 0 + e^{\frac{-2 i \pi}{8} 5 \cdot 1} + 0 - e^{\frac{-2 i \pi}{8} 5 \cdot 3} + 0 + e^{\frac{-2 i \pi}{8} 5 \cdot 5} + 0 - e^{\frac{-2 i \pi}{8} 5 \cdot 7} \\
                                                                           & = 0 + e^{-\frac{i \pi}{4}} + 0 - e^{-\frac{3 i \pi}{4}} + 0 + e^{\frac{3 i \pi}{4}} + 0 - e^{\frac{i \pi}{4}}                                               \\
                                                                           & = 0
                        \end{align*}
                  \item \begin{align*}
                              \sum_{n=0}^{7}{x_n e^{-\frac{2 i \pi}{4}6n}} & = \sum_{n=0}^{7}{x_n e^{-\frac{2 i \pi}{8}6n}}                                                                                                              \\
                                                                           & = 0 + e^{\frac{-2 i \pi}{8} 6 \cdot 1} + 0 - e^{\frac{-2 i \pi}{8} 6 \cdot 3} + 0 + e^{\frac{-2 i \pi}{8} 6 \cdot 5} + 0 - e^{\frac{-2 i \pi}{8} 6 \cdot 7} \\
                                                                           & = 0 + i + 0 + i + 0 + i + 0 + i                                                                                                                             \\
                                                                           & = 0 + 4i
                        \end{align*}
                  \item \begin{align*}
                              \sum_{n=0}^{7}{x_n e^{-\frac{2 i \pi}{4}7n}} & = \sum_{n=0}^{7}{x_n e^{-\frac{2 i \pi}{8}7n}}                                                                                                              \\
                                                                           & = 0 + e^{\frac{-2 i \pi}{8} 7 \cdot 1} + 0 - e^{\frac{-2 i \pi}{8} 7 \cdot 3} + 0 + e^{\frac{-2 i \pi}{8} 7 \cdot 5} + 0 - e^{\frac{-2 i \pi}{8} 7 \cdot 7} \\
                                                                           & = 0 + e^{-\frac{3 i \pi}{4}} + 0 - e^{-\frac{i \pi}{4}} + 0 + e^{\frac{i \pi}{4}} + 0 - e^{\frac{3 i \pi}{4}}                                               \\
                                                                           & = 0
                        \end{align*}
                        \begin{equation*}
                              DFT([0, 1, 0, -1, 0, 1, 0, -1]) = [0, 0, -4i, 0, 0, 0, 4i, 0]
                        \end{equation*}
            \end{enumerate}
            \textbf{Analysis}: Using our imagination, we can see that the data follows a frequency of up, down, up, down. Thus, it makes sense for the output array to have minimal values as it can encapsulate a theoretical sin / cos pattern.
\end{enumerate}

\section*{Problem 2}
Suppose the splits of every level of Quicksort are in \((1 - \alpha, \alpha)\) as a fraction of \(n\), where \(0 < \alpha \leq 1/2\) is a constant.

\subsection*{Minimum Depth}
Given an arbitrary \(\alpha\), the minimum level is
\begin{align*}
      \log_{1/a}n & = \frac{\log_2n}{\log_2 1/a}     \\
                  & = \frac{\log_2 n}{\log_2 a^{-1}} \\
                  & = \frac{log_2 n}{-1 \log a}      \\
                  & = -\frac{\log n}{\log a}
\end{align*}

\subsection*{Maximum Depth}
Similarly, since \(\alpha\) is lower bounded by \(1 - \alpha\), the maximum level is
\begin{align*}
      \log_\frac{1}{a - 1} n & = \frac{\lg n}{\lg \frac{1}{1-a}} \\
                             & = \frac{\lg n}{\lg(1-a) ^ {-1}}   \\
                             & = \frac{\lg n}{-\lg(1-a)}         \\
                             & = -\frac{\lg n}{\lg(1-a)}
\end{align*}
\end{document}