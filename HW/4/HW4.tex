\documentclass{article}

\usepackage[tmargin=0.5in,bmargin=0.25in]{geometry}
\usepackage{amsmath, amssymb, amsthm}
\usepackage{listings}
\usepackage{multicol}
\usepackage{enumerate,enumitem}

\title{CSCI 305 HW 4}
\author{Isaac Boaz}

\begin{document}
\maketitle

\section*{Problem 1}
Show that \[\sum_{i=1}^{n} = \frac{n(n+1)}{2} \text{ for } n \geq 1\]

\begin{proof}
    By induction. \\
    \textbf{Base case}: Observe at \(n = 1\),
    \begin{equation*}
        \sum_{i=1}^{1} = \frac{1(1+1)}{2} = 1
    \end{equation*} holds true. \\
    \textbf{Inductive step}: Suppose \(\sum_{i=1}^{n} = \frac{n(n+1)}{2}\). \\
    That is to say, \(1 + 2 + 3 + \cdots + n = \frac{n(n+1)}{2}\). \\
    Then \(1 + 2 + 3 + \cdots + n + (n + 1) = \frac{n(n+1)}{2} + (n+1)\). \\
    \begin{align*}
        \frac{n(n+1)}{2} + (n+1) & = \frac{n(n+1)}{2} + \frac{2(n+1)}{2} \\
                                 & = \frac{(n+1)(n+2)}{2}
    \end{align*}
    Therefore \(\sum_{i=1}^{n} = \frac{n(n+1)}{2} \text{ for } n \geq 1\).
\end{proof}

\pagebreak

\section*{Problem 2}
Show that \[\sum_{i=0}^{k}{ar^i} = a \left( \frac{1-r^{k+1}}{1-r} \right) \text{ for } k \geq 0\]

\begin{proof}
    By induction. \\
    \textbf{Base case}: Observe at \(k = 0\),
    \begin{equation*}
        \sum_{i=0}^{0}{ar^i} = a \left( \frac{1-r^{0+1}}{1-r} \right) = a \left( \frac{1-r}{1-r} \right) = a
    \end{equation*}
    \textbf{Inductive step}: Suppose \(\sum_{i=0}^{k}{ar^i} = a \left( \frac{1-r^{k+1}}{1-r} \right)\). \\
    That is to say, \(ar^0 + ar^1 + ar^2 + \cdots + ar^k = a \left(\frac{1-r^{k+1}}{1-r}\right)\). \\
    Then \(ar^0 + ar^1 + ar^2 + \cdots + ar^k + ar^{k+1} = a \left(\frac{1-r^{k+1}}{1-r}\right) + ar^{k+1}\)
    \begin{align*}
        a \left(\frac{1-r^{k+1}}{1-r}\right) + ar^{k+1} & = a \left(\frac{1-r^{k+1}}{1-r}\right) + ar^{k+1} \cdot \frac{1-r}{1-r}             \\
                                                        & = a \left(\frac{1-r^{k+1}}{1-r}\right) + a \left(\frac{r^{k+1}-r^{k+2}}{1-r}\right) \\
                                                        & = a \left(\frac{1-r^{k+1}+r^{k+1}-r^{k+2}}{1-r}\right)                              \\
                                                        & = a \left(\frac{1-r^{k+2}}{1-r}\right)
    \end{align*}
    Therefore \(\sum_{i=0}^{k}{ar^i} = a \left( \frac{1-r^{k+1}}{1-r} \right)\).
\end{proof}

\pagebreak

\section*{Problem 3}
Show that the Fibonacci numbers defined recursively by
\begin{align*}
    f(n) & = f(n-1) + f(n-2) \\
    f(0) & = 0, f(1) = 1
\end{align*}
grow as \(f(n) = O(b^n)\) for some \(b > 1\).

\begin{proof}
    By induction. \\
    Note that \(f(n) = O(b^n) \implies f(n) \leq cb^n\). \\
    \textbf{Base case}: Observe at \(n = 0\),
    \begin{align*}
        f(0) = 0 \leq cb^0 \\
        0 \leq 1
    \end{align*}
    and at \(n = 1\)
    \begin{align*}
        f(1) = 1 \leq cb^1 \\
        1 \leq 2
    \end{align*}
    holds true for \(c = 1, b = 2\) \\
    \textbf{Inductive step}: Suppose \(f(n) \leq cb^n\). \\
    Then \begin{align*}
        f(n+1) & = f(n) + f(n-1)      \\
               & \leq cb^n + cb^{n-1} \\
               & \leq cb^{n-1}(b+1)
    \end{align*}
    We need to choose a \(b \text{ s.t. } b + 1 = b^2\).
    \begin{align*}
        b^2         & = b + 1                           \\
        b^2 - b - 1 & = 0                               \\
        b           & = \frac{1 \pm \sqrt{5}}{2} = \phi
    \end{align*}
    Plugging in \(\phi\) for \(b\),
    \begin{align*}
        f(n+1) & \leq c\phi^{n-1}(\phi+1) \\
               & \leq c\phi^{n-1}\phi^2   \\
               & \leq c\phi^{n+1}
    \end{align*}
    Therefore \(f(n+1) \leq cb^n\). \\
\end{proof}

% \pagebreak

% \section*{Problem 4}
% Show that the recursion
% \begin{equation*}
%     T(n) = T(\frac{n}{3}) + T(\frac{2n}{3}) + O(n)
% \end{equation*}
% has a solution which is
% \begin{equation*}
%     T(n) = O(n \log n)
% \end{equation*}

% Note that \(T(n) = O(n \log n) \implies T(n) \leq n \log n\) \\
% \begin{proof}
%     Given \(T(n) = n \log n\)
%     \begin{align*}
%         T\left(\frac{n}{3}\right) + T\left(\frac{2n}{3}\right) + O(n) \leq n \log n \\
%         c \frac{n}{3} \log \frac{n}{3} + c \frac{2n}{3} \log \frac{2n}{3} + O(n) \leq n \log n
%     \end{align*}
%     \begin{align*}
%         \frac{2n}{3} \log \frac{2n}{3} & = \frac{2n}{3}(\log n + \log \frac{2}{3}) \\
%                                        & \leq \frac{2n}{3} \log n
%     \end{align*}
%     \begin{align*}
%         T(n) \leq 
%     \end{align*}
% \end{proof}

\end{document}