\documentclass{article}

\usepackage[tmargin=0.2in,bmargin=0.25in]{geometry}
\usepackage{amsmath, amssymb, amsthm}
\usepackage{listings}
\usepackage{multicol}
\usepackage{enumerate,enumitem}
\usepackage{pgfplots}
\pgfplotsset{compat=1.18}

\title{CSCI 305 HW 3}
\author{Isaac Boaz}

\begin{document}
\maketitle

\subsection*{Problem 1}
Show that \(5n + 12 = O(n)\) directly from the definition.

\begin{proof}
    \(5n + 12 = O(n) \implies 5n + 12 < cn\) \\
    Suppose \(c = 6\), then
    \begin{align*}
        5n + 12 < 6n \\
        12 < n
    \end{align*}
    which shows us that \(5n + 12 = O(n)\) for \(n_0 = 12, c = 6\).
\end{proof}

\subsection*{Problem 2}
Show that \(2n + 100 \log n= \Omega(n)\)
\begin{proof}
    \(2n + 100 \log n = \Omega(n) \implies 2n + 100 \log n > cn\) \\
    Suppose \(c = 2\), then
    \begin{align*}
        2n + 100 \log n > 2n \\
        100 \log n > 0
    \end{align*}
    which holds true for all \(n > 1\), showing us that \(2n + 100 \log n = \Omega(n)\) for \(n_0=1,c=2\).
\end{proof}

\subsection*{Problem 3}
Show that \(\lceil 2n \rceil + 5/n = \Theta(n)\).
\begin{proof}
    \begin{align*}
        \lceil 2n \rceil + 5/n       & = \Theta(n) \implies \\
        \lceil 2n \rceil + 5/n       & = O(n)               \\
        \land \lceil 2n \rceil + 5/n & = \Omega(n)
    \end{align*}
    \subsubsection*{Part 1}
    \(\lceil 2n \rceil + 5/n = O(n) \implies \lceil 2n \rceil + 5/n < cn\) for some \(c\). \\
    Suppose \(c = 3\), then
    \begin{align*}
        \lceil 2n \rceil + 5/n < 3n & \rightarrow \lceil 2n \rceil + 5/n \leq 2n + 1 + 5/n \\
        2n + 1 + 5/n                & < 3n                                                 \\
        1 + 5/n                     & < n
    \end{align*}
    which holds true for all \(n \geq 5\), showing us that \(\lceil 2n \rceil + 5/n = O(n)\) for \(n_0 = 5, c=3\).
    \subsubsection*{Part 2}
    \(\lceil 2n \rceil + 5/n = \Omega(n) \implies \lceil 2n \rceil + 5/n > cn\) for some \(c\). \\
    Suppose \(c = 1\), then
    \begin{align*}
        \lceil 2n \rceil + 5/n > n & \rightarrow \lceil 2n \rceil + 5/n \geq 2n + 1 + 5/n \\
        2n + 1 + 5/n               & > n                                                  \\
        n + \frac{5}{n}            & > -1
    \end{align*}
    which holds true for all \(n \geq 0\), showing us that \(\lceil 2n \rceil + 5/n = \Omega(n)\) for \(n_0 = 0, c=1\).
\end{proof}


\end{document}