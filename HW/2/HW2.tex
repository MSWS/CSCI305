\documentclass{article}

\usepackage[tmargin=0.3in,bmargin=0.25in]{geometry}
\usepackage{amsmath, amssymb, amsthm}
\usepackage{listings}
\usepackage{multicol}
\usepackage{enumerate,enumitem}

\title{CSCI 305 HW 2}
\author{Isaac Boaz}

\begin{document}
\maketitle


\lstset
{ %Formatting for code in appendix
    language=Python,
    basicstyle=\footnotesize,
    numbers=left,
    stepnumber=1,
    showstringspaces=false,
    tabsize=1,
    breaklines=true,
    breakatwhitespace=false,
    numbersep=-30pt,
    xleftmargin=-20pt
}

\setlength{\columnseprule}{0.5pt}

\begin{multicols}{2}
    \null
    \begin{lstlisting}
        for j = 2 to n
            key = A[j]
            // Insert A[j] into the sorted sequence A[1..j-1]
            i = j-1
            while i > 0 and A[i] > key
                A[i+1] = A[i]
                i = i -1
            A[i+1] = key
    \end{lstlisting}
    \columnbreak
    \footnotesize
    \begin{tabular}{lcc}
        cost  & times &                         \\
              & b.c.  & w.c.                    \\
        $C_1$ & $n$   & $n$                     \\
        $C_2$ & $n-1$ & $n-1$                   \\
        $C_3$ & $n-1$ & $n-1$                   \\
        \\
        $C_4$ & $n-1$ & $n-1$                   \\
        $C_5$ & $n$   & $\sum_{i=1}^{j-1}$      \\
        $C_6$ & 0     & $\sum_{i=1}^{j-1}{i-1}$ \\
        $C_7$ & 0     & $\sum_{i=1}^{j-1}{i-1}$ \\
        $C_8$ & $n-1$ & $n-1$
    \end{tabular}
\end{multicols}

\begin{enumerate}
    \item Walk through the algorithm using the following sequences.
          \begin{multicols}{3}
              \begin{center}
                  8 2 4 9 3 6
                  \begin{enumerate}[label=\arabic*.,leftmargin=1cm]
                      \itemsep -0.25em
                      \item 2 8 4 9 3 6
                      \item 2 4 8 9 3 6
                      \item 2 4 8 3 9 6
                      \item 2 4 3 8 9 6
                      \item 2 3 4 8 9 6
                      \item 2 3 4 8 6 9
                      \item 2 3 4 6 8 9
                  \end{enumerate}
              \end{center}
              \vfill\null
              \columnbreak
              \begin{center}
                  2 3 4 6 8 9 \\
                  \textit{(terminates)}
              \end{center}
              \vfill\null
              \columnbreak
              \begin{center}

                  9 8 6 4 3 2
                  \begin{enumerate}[label=\arabic*.]
                      \itemsep -0.25em
                      \item 8 9 6 4 3 2
                      \item 8 6 9 4 3 2
                      \item 6 8 9 4 3 2
                      \item 6 8 4 9 3 2
                      \item 6 4 8 9 3 2
                      \item 4 6 8 9 3 2
                      \item 4 6 8 3 9 2
                      \item 4 6 3 8 9 2
                      \item 4 3 6 8 9 2
                      \item 3 4 6 8 9 2
                      \item 3 4 6 8 2 9
                      \item 3 4 6 2 8 9
                      \item 3 4 2 6 8 9
                      \item 3 2 4 6 8 9
                      \item 2 3 4 6 8 9
                  \end{enumerate}
              \end{center}
          \end{multicols}
    \item Assign a cost to each line. \\
          \[C_1, C_2, \dots, C_8\]
    \item What is the best case scenario? \\
          The best case scenario occurs when the array is already fully sorted.
    \item What is the worse case scenario? \\
          The worse case scenario occurs when the array is reverse-sorted.
    \item Give the best case runtime.
          \begin{equation*}
              \begin{split}
                  [C_1 \cdot n] +
                  [C_2 \cdot (n - 1)] +
                  [C_3 \cdot (n - 1)] +
                  [C_4 \cdot (n - & 1)] +
                  [C_5] \cdot n +
                  [C_6 \cdot 0] +
                  [C_7 \cdot 0] +
                  [C_8 \cdot (n - 1)] \\
                  \implies O(n)
              \end{split}
          \end{equation*}
    \item Give the worst case runtime.
          \begin{equation*}
              \begin{split}
                  [C_1 \cdot n] +
                  [C_2 \cdot (n - 1)] + &
                  [C_3 \cdot (n - 1)] +
                  [C_4 \cdot (n - 1)] + \\
                  [C_5 \cdot \sum_{i=1}^{j-1}{i}] +
                  [C_6 \cdot \sum_{i=1}^{j-1}{(i-1)}] + &
                  [C_6 \cdot \sum_{i=1}^{j-1}{(i-1)}] +
                  [C_8 \cdot (n - 1)] \\
                  \implies & O(n^2)
              \end{split}
          \end{equation*}
\end{enumerate}

\end{document}